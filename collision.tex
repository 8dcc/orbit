\documentclass{amsart}

\title{Elastic collisions}
\author{8dcc}

% Various math utilities, like align*
\usepackage{amsmath}

% Graphs
\usepackage{tikz}

% Hyperlinks
\usepackage[hidelinks]{hyperref}

% Change the spacing between paragraphs
\setlength{\parskip}{\baselineskip}

% Remove author and extra info from the headers
\pagestyle{plain}

% New environment adding spacing for tikz pictures
\newenvironment{tikzpicturecenter}
{\begin{center}\begin{tikzpicture}}
{\end{tikzpicture}\end{center}}

\begin{document}
\maketitle

The distance between two bodies is the hypotenuse of a right triangle whose two
cathetus are the difference between the \(x\) and \(y\) coordinates of the
centers of the two bodies.

% NOTE: Environment defined above
\begin{tikzpicturecenter}
  \draw[thin, gray, dotted] (0,0) grid (8,7);
  \draw[gray] (2,5) circle (1.2) node[above right]{A};
  \draw[gray] (6,2) circle (1.5) node[above right]{B};
  \draw[thick] (2,5) -- (6,2);
  \draw[thick, blue] (2,2) -- (2,5) node[pos=0.5, left]{\(d_y\)};
  \draw[thick, red] (2,2) -- (6,2) node[pos=0.5, below]{\(d_x\)};
\end{tikzpicturecenter}

\begin{align*}
  \text{dx}       &= B_x - A_x            \\
  \text{dy}       &= B_y - A_y            \\
  \text{distance} &= \sqrt{\text{dx}^2 + \text{dx}^2}
\end{align*}

If the sum of the radius of \(A\) and \(B\) is greater or equal than the
distance between the centers, they are colliding.

\iffalse

The gravitational force of each body is calculated with the following formula.

\begin{displaymath}
  F = G \frac{m_1m_2}{r^2}
\end{displaymath}

Where \(G\) is the gravitational constant, \(m_1\) and \(m_2\) are the mass of
each body, and \(r\) is the distance between the objects.

The effect of a force is to accelerate the body. The relationship is the
following.

\begin{displaymath}
  F = m a
\end{displaymath}

Where \(F\) is the force, \(m\) is the mass and \(a\) is the acceleration of
the body. Therefore, to get the acceleration from the force, we can do the
following.

\begin{displaymath}
  a = \frac{F}{m}
\end{displaymath}

The force has a direction. It acts towards the direction of the line joining
the centres of the two bodies. We can get the X and Y coordinates of the
acceleration with some trigonometry.

\begin{align*}
  a_x &= a \cos(\theta) \\
  a_y &= a \sin(\theta) \\
\end{align*}

Where \(a_x\) and \(a_y\) are the X and Y accelerations, \(a\) is the
acceleration, and \(\theta\) is the angle that the line joining the centers make
with the horizontal.

\fi

\begin{thebibliography}{9}
\bibitem{elastic}
  \url{https://en.wikipedia.org/wiki/Elastic_collision}
\end{thebibliography}

\end{document}
