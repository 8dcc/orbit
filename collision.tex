\documentclass{amsart}

\title{Elastic collisions}
\author{8dcc}

% Various math utilities, like align*
\usepackage{amsmath}

% Graphs
\usepackage{tikz}

% Code highlighting
\usepackage{listings}

% Hyperlinks
\usepackage[hidelinks]{hyperref}

% Change the spacing between paragraphs
%\setlength{\parskip}{\baselineskip}

% Remove author and extra info from the headers
\pagestyle{plain}

% New environment adding spacing for tikz pictures
\newenvironment{tikzpicturecenter}
{\begin{center}\begin{tikzpicture}}
{\end{tikzpicture}\end{center}}

\begin{document}
\maketitle

This document will try to explain the math behind the collision simulations. The
simulations will progressively get more complex. For the simplest case, we are
going to simulate an elastic collision between a body $A$ and a immobile body
$B$.

Since it's an elastic collision, the kinetic energy will remain the same after
the impact. For now, we will ignore the masses and focus on the impact and the
reflection itself.

\section{Distance between the bodies}

The distance between two bodies can be calculated with the hypotenuse of a right
triangle whose two cathetus are the difference between the $x$ and $y$
coordinates of each center.

% NOTE: Environment defined above
\begin{tikzpicturecenter}
  % Coordinates of circles
  \coordinate (a) at (2,5);
  \coordinate (b) at (6,2);

  % Grid
  \draw[thin, gray, dotted] (0,0) grid (8,7);

  % Circles
  \draw (a) circle (1.2);
  \draw (b) circle (1.5);

  % Lines
  \draw[thick] (a) -- (b) node[pos=0.5, above right]{$d$};
  \draw[thick, blue] (2,2) -- (a) node[pos=0.5, left]{$\Delta_y$};
  \draw[thick, red] (2,2) -- (b) node[pos=0.5, below]{$\Delta_x$};

  % Points in centers
  \filldraw (a) circle (1pt) node[above right]{$A$};
  \filldraw (b) circle (1pt) node[above right]{$B$};
\end{tikzpicturecenter}

\begin{displaymath}
  d = \sqrt{(B_x - A_x)^2 + (B_y - A_y)^2}
\end{displaymath}

In C code:

\begin{lstlisting}[language=C]
  float dx = b.x - a.x;
  float dy = b.y - a.y;
  float distance = sqrtf(dx * dx + dy * dy);
\end{lstlisting}

\newpage

\section{Bouncing off another immobile body}

If the sum of the radius of $A$ and $B$ is greater or equal than the distance
between the centers, they are colliding. Since this is an elastic collision, and
since the second body is immobile, the velocity of the first body will remain
the same, but its direction will change.

\subsection{Normalization}

By normalizing a vector, we can determine its direction. To get the normalized
or unitary vector, we can use this formula:

\begin{displaymath}
  \hat{u} = \frac{u}{\|u\|}
\end{displaymath}

Where $u$ is the vector, $\hat{u}$ is the normalized vector, and $\|u\|$ is the
norm or magnitude of the vector.

We normalize our impact vector (the line between the centers) by dividing
$\Delta_x$ and $\Delta_y$ by the length of the vector, which we calculated
earlier. This normalized vector will be used to determine the direction of the
impact.

\begin{align*}
  n_x &= \frac{\Delta_x}{d} \\
  n_y &= \frac{\Delta_y}{d}
\end{align*}

\subsection{Dot product}

We need to determine the magnitude of the velocity of $A$ that is going towards
the center of $B$. We can get a scalar that represents this magnitude with the
dot product.

The dot product or scalar product takes two vectors and returns a scalar that
represents the projection of one vector onto the other. In simpler terms, it's a
way of quantifying how aligned is vector $A$ with vector $B$.

If the dot product is positive, $A$ has a component in the same direction as
$B$. If the dot product is zero, $A$ and $B$ are perpendicular. If it's
negative, $A$ has a component in the opposite direction of $B$.

\begin{tikzpicturecenter}
  \draw[thin, gray, dotted] (0,0) grid (13,4);

  \draw[<->] (2,3) -- (1,1) -- (3,1.5);
  \node at (2,3) [above] {A};
  \node at (3,1.5) [above] {B};
  \node at (2,0.5) [draw, rectangle] {$A \cdot B > 0$};

  \draw[<->] (4.5,2.5) -- (6,1) -- (7.5,2.5);
  \node at (4.5,2.5) [above] {A};
  \node at (7.5,2.5) [above] {B};
  \node at (6,0.5) [draw, rectangle] {$A \cdot B = 0$};

  \draw[<->] (12,1) -- (10,1) -- (9,3);
  \node at (9,3) [above] {A};
  \node at (12,1) [above] {B};
  \node at (10,0.5) [draw, rectangle] {$A \cdot B < 0$};
\end{tikzpicturecenter}

If both vectors are normalized, the returned value will always be in the range
$[-1,1]$.

\newpage

For example, if $V$ represents the velocity vector of body $B$, and $n$
represents the normalized collision vector we calculated earlier:

\begin{tikzpicturecenter}
  % Coordinates
  \coordinate (a) at (2,3);
  \coordinate (b) at (5,2);

  % Grid
  \draw[thin, gray, dotted] (0,0) grid (7,5);

  % Circles. The radius sum the distance between the centers (~3.16)
  \draw (a) circle (1.5);
  \draw (b) circle (1.65);

  % Red and blue arrows
  \draw[->, red] (b) -- (4,2.33) node[pos=0.5, below]{$n$};
  \draw[->, blue] (b) -- (4.5,4.5) node[pos=0.5, right]{$V$};

  % Points in centers
  \filldraw (a) circle (1pt) node[above right]{$A$};
  \filldraw (b) circle (1pt) node[below right]{$B$};
\end{tikzpicturecenter}

Then the dot product would be calculated by:

\begin{displaymath}
  V \cdot n = V_x n_x + V_y n_y
\end{displaymath}

The dot product has a direct relationship with the angle between the two
vectors. It can also be calculated with this formula:

\begin{displaymath}
  a \cdot b = \|a\| \|b\| \cos \theta
\end{displaymath}

Therefore, we can calculate the angle with this formula:

\begin{align*}
  \cos \theta = \frac{a \cdot b}{\|a\| \|b\|} \\
  \theta = \cos^{-1}(\frac{a \cdot b}{\|a\| \|b\|})
\end{align*}

In C code:

\begin{lstlisting}[language=C]
  // Normalize collision vector
  float nx = dx / distance;
  float ny = dy / distance;
  float dot_product = a.vx * nx + a.vy * ny;
  // TODO
\end{lstlisting}

\newpage

% TODO: Last section
\section{Gravitational force}

The gravitational force of each body is calculated with the following formula.

\begin{displaymath}
  F = G \frac{m_1m_2}{r^2}
\end{displaymath}

Where $G$ is the gravitational constant, $m_1$ and $m_2$ are the mass of each
body, and $r$ is the distance between the objects.

The effect of a force is to accelerate the body. The relationship is the
following.

\begin{displaymath}
  F = m a
\end{displaymath}

Where $F$ is the force, $m$ is the mass and $a$ is the acceleration of the
body. Therefore, to get the acceleration from the force, we can do the
following.

\begin{displaymath}
  a = \frac{F}{m}
\end{displaymath}

The force has a direction. It acts towards the direction of the line joining
the centres of the two bodies. We can get the X and Y coordinates of the
acceleration with some trigonometry.

\begin{align*}
  a_x &= a \cos(\theta) \\
  a_y &= a \sin(\theta) \\
\end{align*}

Where $a_x$ and $a_y$ are the X and Y accelerations, $a$ is the acceleration,
and $\theta$ is the angle that the line joining the centers make with the
horizontal.

\newpage

\begin{thebibliography}{9}
\bibitem{elastic}
  \url{https://en.wikipedia.org/wiki/Elastic_collision}
\bibitem{unitvector}
  \url{https://en.wikipedia.org/wiki/Unit_vector}
\bibitem{scalar}
  \url{https://en.wikipedia.org/wiki/Scalar_(mathematics)}
\bibitem{dotproduct}
  \url{https://en.wikipedia.org/wiki/Dot_product}
\bibitem{godot}
  \url{https://docs.godotengine.org/en/stable/tutorials/math/vector_math.html}
\end{thebibliography}

\end{document}
