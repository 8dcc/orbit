\documentclass{amsart}

\title{Elastic collisions}
\author{8dcc}

% Various math utilities, like align*
\usepackage{amsmath}

% Graphs
\usepackage{tikz}

% Code highlighting
\usepackage{listings}

% Hyperlinks
\usepackage[hidelinks]{hyperref}

% Change the spacing between paragraphs
\setlength{\parskip}{\baselineskip}

% Remove author and extra info from the headers
\pagestyle{plain}

% New environment adding spacing for tikz pictures
\newenvironment{tikzpicturecenter}
{\begin{center}\begin{tikzpicture}}
{\end{tikzpicture}\end{center}}

\begin{document}
\maketitle

First, we are going to simulate an elastic collision between two bodies. The
body \(A\) will be moving towards the body \(B\) but this one will be static.

Since it's an elastic collision, the kinetic energy will remain the same after
the impact. We will ignore the masses, and only are about the impact and the
reflection itself.

\section{Distance between the bodies}

The distance between two bodies can be calculated with the hypotenuse of a right
triangle whose two cathetus are the difference between the \(x\) and \(y\)
coordinates of each center.

% NOTE: Environment defined above
\begin{tikzpicturecenter}
  \draw[thin, gray, dotted] (0,0) grid (8,7);
  \draw[gray] (2,5) circle (1.2) node[above right]{A};
  \draw[gray] (6,2) circle (1.5) node[above right]{B};
  \draw[thick] (2,5) -- (6,2);
  \draw[thick, blue] (2,2) -- (2,5) node[pos=0.5, left]{\(d_y\)};
  \draw[thick, red] (2,2) -- (6,2) node[pos=0.5, below]{\(d_x\)};
\end{tikzpicturecenter}

\begin{displaymath}
  d = \sqrt{(B_x - A_x)^2 + (B_y - A_y)^2}
\end{displaymath}

\newpage

In C code:

\begin{lstlisting}[language=C]
  float dx = b.x - a.x;
  float dy = b.y - a.y;
  float distance = sqrtf(dx * dx + dy * dy);
\end{lstlisting}

\section{Bouncing off another static body}

If the sum of the radius of \(A\) and \(B\) is greater or equal than the
distance between the centers, they are colliding. Since this is an elastic
collision, and since the second body is static, the velocity of the first body
will remain the same, but the direction will change.

% TODO

\iffalse
\section{Gravitational force}
% TODO: Last section

The gravitational force of each body is calculated with the following formula.

\begin{displaymath}
  F = G \frac{m_1m_2}{r^2}
\end{displaymath}

Where \(G\) is the gravitational constant, \(m_1\) and \(m_2\) are the mass of
each body, and \(r\) is the distance between the objects.

The effect of a force is to accelerate the body. The relationship is the
following.

\begin{displaymath}
  F = m a
\end{displaymath}

Where \(F\) is the force, \(m\) is the mass and \(a\) is the acceleration of
the body. Therefore, to get the acceleration from the force, we can do the
following.

\begin{displaymath}
  a = \frac{F}{m}
\end{displaymath}

The force has a direction. It acts towards the direction of the line joining
the centres of the two bodies. We can get the X and Y coordinates of the
acceleration with some trigonometry.

\begin{align*}
  a_x &= a \cos(\theta) \\
  a_y &= a \sin(\theta) \\
\end{align*}

Where \(a_x\) and \(a_y\) are the X and Y accelerations, \(a\) is the
acceleration, and \(\theta\) is the angle that the line joining the centers make
with the horizontal.

\fi

\begin{thebibliography}{9}
\bibitem{elastic}
  \url{https://en.wikipedia.org/wiki/Elastic_collision}
\end{thebibliography}

\end{document}
