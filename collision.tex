\documentclass{amsart}

\title{Elastic collisions}
\author{8dcc}

% ------------------------------------------------------------------------------
% Packages
% ------------------------------------------------------------------------------

% Hyperlinks
\usepackage[hidelinks]{hyperref}

% Slightly smaller margins
\usepackage[a4paper,
            left=4cm,
            right=4cm,
            top=3cm,
            bottom=3cm,
            footskip=1cm]{geometry}

% Math
\usepackage{amsmath}

% Graphs
\usepackage{tikz}
% Use Calc library for coordinate calculations
\usetikzlibrary{calc}
% Use Decorations for drawing braces
\usetikzlibrary{decorations.pathreplacing}
% Change default arrow style
\tikzset{>=stealth}

% New environment adding spacing for TikZ pictures
\newenvironment{tikzpicturecenter}
{\begin{center}\begin{tikzpicture}}
{\end{tikzpicture}\end{center}}

% Code highlighting.
\usepackage{listings}
\lstset{
  % Showing spaces
  showspaces=false,
  showstringspaces=false,
  showtabs=false,
  % Indentation and breaks
  tabsize=4,
  breaklines=true,
  breakatwhitespace=true,
  columns=flexible,
  % Show left, right, top and bottom borders
  frame=tblr,
  % Misc
  aboveskip=3mm,
  belowskip=3mm,
  basicstyle={\small\ttfamily},
}

% Different monospace font for code blocks (listings)
\usepackage{inconsolata}

% Remove author and extra info from the headers
\pagestyle{plain}

% ------------------------------------------------------------------------------
% Document start
% ------------------------------------------------------------------------------
\begin{document}

\maketitle

This document will try to explain the math behind my collision simulations. The
simulations will progressively get more complex. For the simplest case, we are
going to simulate an elastic collision between a body $A$ and a immobile body
$B$.

Since it's an elastic collision, the kinetic energy will remain the same after
the impact. For now, we will ignore the masses and focus on the impact and the
reflection itself.

% ------------------------------------------------------------------------------
\section{Distance between the bodies}
% ------------------------------------------------------------------------------

The distance between two bodies can be calculated with the hypotenuse of a right
triangle whose two cathetus are the difference between the $x$ and $y$
coordinates of each center. In other words, the distance between $A$ and $B$ is
the magitude of the vector $\vec{AB}$, noted as $\|\vec{AB}\|$.

% NOTE: Environment defined above
\begin{tikzpicturecenter}
  % Coordinates of circles
  \coordinate (a) at (2,5);
  \coordinate (b) at (6,2);

  % Grid
  \draw[thin, gray, dotted] (0,0) grid (8,7);

  % Circles
  \draw (a) circle (1.2);
  \draw (b) circle (1.5);

  % Lines
  \draw[thick] (a) -- (b) node[pos=0.5, above right]{$m$};
  \draw[thick, blue] (2,2) -- (a) node[pos=0.5, left]{$\Delta_y$};
  \draw[thick, red] (2,2) -- (b) node[pos=0.5, below]{$\Delta_x$};

  % Points in centers
  \filldraw (a) circle (1pt) node[above right]{A};
  \filldraw (b) circle (1pt) node[above right]{B};
\end{tikzpicturecenter}

\begin{displaymath}
  m = \sqrt{(B_x - A_x)^2 + (B_y - A_y)^2}
\end{displaymath}

\bigskip

In C code:

\begin{lstlisting}[language=C]
  float dx = b.x - a.x;
  float dy = b.y - a.y;
  float distance = sqrtf(dx * dx + dy * dy);
\end{lstlisting}

% ------------------------------------------------------------------------------
\section{Bouncing off another immobile body}
% ------------------------------------------------------------------------------

If the sum of the radius of $A$ and $B$ is greater or equal than the distance
between the centers, they are colliding. Since this is an elastic collision, and
since the second body is immobile, the velocity of the first body will remain
the same, but its direction will change.

% ------------------------------------------------------------------------------
\subsection{Normalization}
% ------------------------------------------------------------------------------

By normalizing a vector, we can determine its direction. To get the normalized
or unitary vector, we can use this formula:

\begin{displaymath}
  \hat{u} = \frac{v}{\|v\|}
\end{displaymath}

Where $v$ is the vector, $\|v\|$ is its norm or magnitude, and $\hat{u}$ is the
normalized vector.

To normalize our impact vector (the line between the centers), we divide
$\Delta_x$ and $\Delta_y$ by the length of the vector, which we calculated
earlier. This normalized vector will be used to determine the direction of the
impact.

\begin{align*}
  n_x &= \frac{\Delta_x}{d} \\
  n_y &= \frac{\Delta_y}{d}
\end{align*}

In the next figure, $n$ represents the normalized impact vector, although its
magnitude might not be very accurate for this specific example.

\begin{tikzpicturecenter}
  % Coordinates
  \coordinate (a) at (5,2);
  \coordinate (b) at (2,3);

  % Grid
  \draw[thin, gray, dotted] (0,0) grid (7,5);

  % Circles. The radius sum the distance between the centers (~3.16)
  \draw (a) circle (1.65);
  \draw (b) circle (1.5);

  % Red and blue arrows
  \draw[red] (a) -- (b) node[pos=0.75, below]{$\vec{AB}$};
  \draw[black, ->, dashed] (a) -- (4,2.333) node[pos=0.5, below]{$n$};

  % Points in centers
  \filldraw (a) circle (1pt) node[below right]{$A$};
  \filldraw (b) circle (1pt) node[above right]{$B$};
\end{tikzpicturecenter}

% ------------------------------------------------------------------------------
\subsection{Dot product}
% ------------------------------------------------------------------------------

We need to determine the magnitude of the velocity of $A$ that is going towards
the center of $B$. We can get a scalar that represents this magnitude with the
dot product.

The dot product or scalar product takes two vectors and returns a scalar that
represents the projection of one vector onto the other. In simpler terms, it's a
way of quantifying how aligned is vector $a$ with vector $b$.

The formula is the following:

\begin{displaymath}
  a \cdot b = a_x b_x + a_y b_y
\end{displaymath}

If both vectors are normalized, the returned value will always be in the range
$[-1,1]$.

The dot product has a direct relationship with the angle formed by the two
vectors. The dot product of two \textbf{unit vectors} is the cosine of the
angle.

\begin{displaymath}
  \hat{a} \cdot \hat{b} = \cos \theta
\end{displaymath}

To calculate the dot product of non-normalized vectors, this formula is used:

\begin{displaymath}
  a \cdot b = \|a\| \|b\| \cos \theta
\end{displaymath}

You can think of the dot product as the shadow that $a$ projects over $b$.

\begin{tikzpicturecenter}
  \pgfmathsetmacro{\Radius}{2}
  \pgfmathsetmacro{\ArcRadius}{1}

  \pgfmathsetmacro{\ArcAngleBeta}{40}

  \pgfmathsetmacro{\BetaX}{\ArcRadius*cos(\ArcAngleBeta)}
  \pgfmathsetmacro{\BetaY}{\ArcRadius*sin(\ArcAngleBeta)}

  \pgfmathsetmacro{\LabelX}{\ArcRadius*cos(\ArcAngleBeta/2)}
  \pgfmathsetmacro{\LabelY}{\ArcRadius*sin(\ArcAngleBeta/2)}

  \coordinate (o) at (0,0);
  \coordinate (ub) at (\BetaX, \BetaY);
  \coordinate (a1) at (\Radius,0);
  \coordinate (b1) at ($\Radius*(ub)$);
  \coordinate (a2) at ($1.7*(a1)$);
  \coordinate (b2) at ($1.7*(b1)$);
  \coordinate (b1shadow) at ($(\Radius*\BetaX, 0)$);
  \coordinate (b2shadow) at ($(1.7*\Radius*\BetaX, 0)$);

  % The purple arc
  \filldraw[violet, fill opacity=0.2] (o) -- (\ArcRadius,0) arc (0:\ArcAngleBeta:\ArcRadius) -- cycle;
  \node at (\LabelX, \LabelY) [left] {$\theta$};

  % Dotted circumference
  \draw[dotted] (o) circle (\Radius);

  % Sine dashed lines ("shadows")
  \draw[dashed, gray] (b1) -- (b1shadow);
  \draw[dashed, gray] (b2) -- (b2shadow);

  % Gray A and B lines
  \draw[->, thick, gray] (o) -- (a2) node[right]{$\vec{a}$};
  \draw[->, thick, gray] (o) -- (b2) node[yshift=0.2cm, xshift=0.2cm]{$\vec{b}$};

  % Black UA and UB lines and dots
  \draw[] (o) -- (a1);
  \filldraw (a1) circle (1pt) node[above right]{$\hat{a}$};
  \draw[] (o) -- (b1) node[yshift=0.3cm, xshift=-0.3cm, pos=0.5]{$r = 1$};
  \filldraw (b1) circle (1pt) node[yshift=0.4cm]{$\hat{b}$};

  % Brace to indicate the cosine of the normalized vectors
  \draw[decorate, decoration={brace,amplitude=5pt,mirror}] (o) -- (b1shadow)
  node [midway, yshift=-0.35cm] {$\hat{a} \cdot \hat{b} = \cos(\theta)$};

  % Brace to indicate the cosine of the non-normalized vectors
  \draw[decorate, decoration={brace,amplitude=5pt,mirror,raise=0.7cm}] (o) -- (b2shadow)
  node [midway, yshift=-1.05cm] {$\vec{a} \cdot \vec{b}$};

  % Center of the circumference
  \filldraw (o) circle (1pt);
\end{tikzpicturecenter}

For a more detailed and interactive explanation of the dot product, see Math
Insight \cite{dotproduct_mathinsight}.

With this in mind, the dot product can be used to calculate the angle itself,
although this won't be used for the simulation.

\begin{align*}
  \cos \theta &= \frac{a \cdot b}{\|a\| \|b\|} \\
  \theta &= \cos^{-1} \left( \frac{a \cdot b}{\|a\| \|b\|} \right)
\end{align*}

We can get a lot of information from the dot product. If the dot product is
positive, $a$ has a component in the same direction as $b$. If the dot product
is zero, $a$ and $b$ are perpendicular. If it's negative, $a$ has a component in
the opposite direction of $b$.

\begin{tikzpicturecenter}
  \draw[thin, gray, dotted] (0,0) grid (13,4);

  \draw[<->] (2,3) -- (1,1) -- (3,1.5);
  \node at (2,3) [above] {$a$};
  \node at (3,1.5) [above] {$b$};
  \node at (2,0.5) [draw, rectangle] {$a \cdot b > 0$};

  \draw[<->] (4.5,2.5) -- (6,1) -- (7.5,2.5);
  \node at (4.5,2.5) [above] {$a$};
  \node at (7.5,2.5) [above] {$b$};
  \node at (6,0.5) [draw, rectangle] {$a \cdot b = 0$};

  \draw[<->] (12,1) -- (10,1) -- (9,3);
  \node at (9,3) [above] {$a$};
  \node at (12,1) [above] {$b$};
  \node at (10,0.5) [draw, rectangle] {$a \cdot b < 0$};
\end{tikzpicturecenter}

% ------------------------------------------------------------------------------
\subsection{Perpendicular velocity}
% ------------------------------------------------------------------------------

After calculating the dot product of $v$ (the velocity vector of body $A$) and
$n$ (the normalized vector of $\vec{AB}$), we can multiply it by $n$ to obtain a
vector with the direction of $n$, but the magnitude of the dot product.

\begin{displaymath}
  n' = \left( v \cdot n \right) n
\end{displaymath}

In the next graph, the dot product of $n$ and $v$ could be, for example, 1.25,
and we multiplied that value by $n$ to obtain $n'$.

\begin{tikzpicturecenter}
  % Coordinates
  \coordinate (a) at (5,2);
  \coordinate (b) at (2,3);

  % Grid
  \draw[thin, gray, dotted] (0,0) grid (7,5);

  % Circles. The radius sum the distance between the centers (~3.16)
  \draw (a) circle (1.65);
  \draw (b) circle (1.5);

  % Red and blue arrows
  \draw[red] (a) -- (b) node[pos=0.75, below]{$\vec{AB}$};
  \draw[black, ->, dashed] (a) -- (3.75,2.415) node[pos=0.5, below]{$n'$};
  \draw[blue, ->] (a) -- (4.5,4.5) node[pos=0.5, right]{$V$};

  \draw[->] (a) -- ($(a)+(1.9,-1.7)$);

  % Points in centers
  \filldraw (a) circle (1pt) node[below right]{$A$};
  \filldraw (b) circle (1pt) node[above right]{$B$};
\end{tikzpicturecenter}

Now that we have quantified how ``direct'' the impact was, we can use this $n'$
vector to calculate how $A$ would bounce off $B$ in an elastic collision.

\begin{displaymath}
  v' = v - 2 n'
\end{displaymath}

% TODO
However, I am not entirely sure why we need to multiply $n'$ by two
(i.e. subtract it twice from $v$).

\bigskip

In C code:

\begin{lstlisting}[language=C]
  float nx = dx / distance;
  float ny = dy / distance;

  float dot_product = b.vx * nx + b.vy * ny;
  float nvx = dot_product * nx;
  float nvy = dot_product * ny;

  float perpendicular_x = b.vx - nvx;
  float perpendicular_y = b.vy - nvy;
  float final_vx = perpendicular_x - nvx;
  float final_vy = perpendicular_y - nvy;
\end{lstlisting}

% ------------------------------------------------------------------------------
\section{Gravitational force}
% ------------------------------------------------------------------------------
% TODO: Last section

The gravitational force of each body is calculated with the following formula.

\begin{displaymath}
  F = G \frac{m_1m_2}{r^2}
\end{displaymath}

Where $G$ is the gravitational constant, $m_1$ and $m_2$ are the mass of each
body, and $r$ is the distance between the objects.

The effect of a force is to accelerate the body. The relationship is the
following.

\begin{displaymath}
  F = m a
\end{displaymath}

Where $F$ is the force, $m$ is the mass and $a$ is the acceleration of the
body. Therefore, to get the acceleration from the force, we can do the
following.

\begin{displaymath}
  a = \frac{F}{m}
\end{displaymath}

The force has a direction. It acts towards the direction of the line joining
the centres of the two bodies. We can get the X and Y coordinates of the
acceleration with some trigonometry.

\begin{align*}
  a_x &= a \cos(\theta) \\
  a_y &= a \sin(\theta) \\
\end{align*}

Where $a_x$ and $a_y$ are the X and Y accelerations, $a$ is the acceleration,
and $\theta$ is the angle that the line joining the centers make with the
horizontal.

% ------------------------------------------------------------------------------
% Bibliography
% ------------------------------------------------------------------------------

\begin{thebibliography}{9}
\bibitem{elastic}
  Wikipedia. \textit{Elastic collision}. Retrieved 23 May 2024, from
  \url{https://en.wikipedia.org/wiki/Elastic_collision}
\bibitem{vectorintro}
  Frank D and Nykamp DQ. \textit{An introduction to vectors}. From Math Insight. Retrieved 23 May 2024, from
  \url{http://mathinsight.org/vector_introduction}
\bibitem{unitvector}
  Wikipedia. \textit{Unit vector}. Retrieved 23 May 2024, from
  \url{https://en.wikipedia.org/wiki/Unit_vector}
\bibitem{scalar}
  Wikipedia. \textit{Scalar}. Retrieved 23 May 2024, from
  \url{https://en.wikipedia.org/wiki/Scalar_(mathematics)}
\bibitem{dotproduct_wikipedia}
  Wikipedia. \textit{Dot product}. Retrieved 23 May 2024, from
  \url{https://en.wikipedia.org/wiki/Dot_product}
\bibitem{dotproduct_mathinsight}
  Nykamp DQ. \textit{The dot product}. From Math Insight. Retrieved 23 May 2024, from
  \url{https://mathinsight.org/dot_product}
\bibitem{godot}
  Godot Engine 4.2 documentation. \textit{Vector math}. Retrieved 23 May 2024, from
  \url{https://docs.godotengine.org/en/stable/tutorials/math/vector_math.html}
\bibitem{realworldphysics}
  Franco Normani. \textit{The Physics Of Billiards}. Retrieved 23 May 2024, from
  \url{https://www.real-world-physics-problems.com/physics-of-billiards.html}
\end{thebibliography}

\end{document}
